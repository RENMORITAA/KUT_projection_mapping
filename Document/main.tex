\documentclass{jlreq}
\usepackage{float}
\usepackage{graphicx}
\usepackage{multicol}
\usepackage{ascmac}
\usepackage{siunitx}
\usepackage{listings}
\usepackage{xcolor}
\usepackage{amsmath}
\usepackage{fancyvrb}
\usepackage{hyperref}

% --- hyperref設定(赤枠防止) ---
\hypersetup{
  colorlinks=true,
  linkcolor=black,
  filecolor=magenta,
  urlcolor=blue,
  citecolor=black,
}

% --- コード表示環境 ---
\lstnewenvironment{mylisting}[1][]%
  {\lstset{
    frame=single,
    basicstyle=\ttfamily\small,
    numbersep=6pt,
    tabsize=3,
    extendedchars=true,
    xleftmargin=17pt,
    framexleftmargin=17pt,
    breaklines=true,
    numbers=left,
    language=Python,  % 必要に応じて変更 (After EffectsやTouchDesignerなら "none" でもOK)
    #1
  }}{}

\begin{document}

% --- 表紙 ---
\begin{titlepage}
  \centering
  \vspace{2cm}
  {\LARGE \bfseries プロジェクションマッピング \par}
  \vspace{3cm}
  {\Large 第1版 \par}
  \vspace{4cm}
  {\Large \today \par}
  \vspace{0.5cm}
  {\LARGE \bfseries 高知工科大学 情報学群 \par}
  \vfill
\end{titlepage}

\newpage
\tableofcontents
\newpage

% --- 各章構成 ---
\section{はじめに}
本プロジェクトでは、高知工科大学のシンボルである時計塔を用いてイルミネーションに合わせて
高知工科大学の冬を盛り上げる目的で開始した。
本レポートでは、企画から設営、映像の制作、運用までの流れについて報告する。

\input{sections/system}
\section{投影環境と設営}

\section{映像制作}

\input{sections/mapping}
\section{運用・当日準備}

\section{成果と改善点}

\input{sections/appendix}


\end{document}
